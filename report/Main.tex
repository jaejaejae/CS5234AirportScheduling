\documentclass[11pt,a4paper]{article}
\usepackage{cite}
\usepackage[utf8]{inputenc}
\usepackage{algorithm}
\usepackage{algorithmic}
\usepackage{amsmath}
\usepackage{amsfonts}
\usepackage{amssymb}
\usepackage{amsthm}

\newtheorem{theorem}{Theorem}[section]
\newtheorem{corollary}{Corollary}[theorem]
\newtheorem{lemma}[theorem]{Lemma}
 
\author{Duong Thanh Dat (A0119656), Ramon Bespinyowong (A0088687)}
\title{Mini-Project 3: Airport Scheduling}



\begin{document}

\maketitle

%\begin{abstract}
%\end{abstract}


\section{Overview}
\label{sec:over}

\subsection{Introduction}
\label{subesc:over-intr}
Gate scheduling for flights is a key operation at airports and is an important problem. Due to globalization, more people are travelling around the world and hence more air transportations are used. Each aircraft is assigned to a terminal, called gates. To handle increasing in traffic volume, we have to assign planes to gates efficiently so that the airport can take more aircraft or more passengers. \par
In this assignment, we study the airport scheduling problem. Given schedules of planes landing and taking-off at the airport, we are required to assign each flight to a gate based on the schedules. The number of required gates must be minimum. \par
There are four problems of airport scheduling problems in this work. The first problem assumes that each plane takes a point of time to land and take off at the airport. The second problem assumes that each plane task has a starting time and an ending time. The third problem studies about the effect of delay in the algorithm used in the first two problems. Finally, the last problem assumes that reallocating planes to new gates is so expensive that we have to design such an algorithm that minimize the reassignments. In other words, we must create such an initial assignment that leaves many buffer time in the schedule as much as possible.  \par
In this work, we purpose algorithms to solve all the problems in Subsections~\ref{subsec:algo-prob1}, ~\ref{subsec:algo-prob2}, ~\ref{subsec:algo-prob3}, and ~\ref{subsec:algo-prob4} respectively. \par
The rest of the work is organized as follows: In the rest of Section~\ref{sec:over} we provide the background and an overview of the related work. Section~\ref{sec:algo} presents algorithms to all four problems of airport scheduling problems. In Section~\ref{sec:expe} we present a thorough experimental study of our algorithms. Finally, Section~\ref{sec:conc} concludes our work and states future plans. \par

\subsection{Preliminaries and Related Work}
\label{subsec:over-prel}
Let $G=\{g_1, ..., g_{|G|}\}$ be a set of gates and $A=\{a_1, ..., a_{|A|}\}$ be a set of planes. Each plane, $a_i$, needs to use any gate from $s_i$ to $f_i$. We need to find optimal solution which uses the fewest number of gates and all the planes have an allocation to any gate in the airport from $s_i$ to $f_i$. In addition, each gate can only take one plane for a particular time. \par
In this Subsection, we are going to look at similar problems to airport scheduling problems.

\subsubsection{Interval Scheduling}
Interval scheduling problems are also known as \textit{fixed job scheduling} or \textit{k-track assignment} problems~\cite{Lenstra2006}. They are different from other scheduling problems in such a way that the starting time of each job is fixed. \par
The formal problem description is given as follows: \textit{n} interval \textit{jobs}, $A = \{a_1, ..., a_{|A|}\}$, in the form [$s_j$, $f_j$) with $0 \leq s_j < f_j$ for $j = 1, ..., n$. These jobs must be processed uninterruptedly. There are \textit{machines}, $M=\{m_1, ..., m_{|M|}\}$. each of which can process at most one job at a time and is always available from $t \in [0, \infty)$. A \textit{schedule} is an assignment of some of the jobs to the machines. \par
We can view airport scheduling problems as interval scheduling problems. Let each gate be a machine and each plane be a job. The time needed at the gate is the process time of each job. \par
There exists several algorithms for solving this algorithm. The lower bound for the number of machines required for processing all the jobs is equal to the largest size of the a subset of jobs that pairwise overlap. Dilworth's chain decomposition is used to show that this lower bound suffices to process all jobs~\cite{Lenstra2006}. Ford and Fulkerson~\cite{FordL.r.1962} can solve this in $O(n^2)$ by using \textit{staircase rule} based on Dilworth's theorem. Alternative procedures are purposed by Hashimoto and Stevens~\cite{hashimoto1971wire} and Gertsbakh and Stern~\cite{gertsbakh1978minimal}. An $O(nlogn)$ procedure is proposed by Gupta et al~\cite{gupta1979optimal}, which will be further explained in Subsection~\ref{subsec:algo-prob2}. There even exists an $O(log n)$ algorithm with $O(n^2/log n)$ processors by Dekel and Sahni~\cite{dekel1983parallel}. \par
%There is a variant of interval scheduling problems called \textit{Interval Scheduling Maximization} which is to find a largest set of non-overlapping jobs to fit one machine. We can get optimal solution by greedy algorithm shown in Algorithm~\ref{algo:greedy}. This algorithm firstly sorts all the jobs based on earliest end time. It is interesting that sorting by the lengths of intervals, by the number of conflicts, or by the starting times does not give optimal solution. Then, the algorithm greedily takes the job with the earliest end time and remove every other jobs that has a conflict with this job. The algorithm continues picking the next earliest jobs until all the jobs are exhausted. The time complexity of this algorithm is $O(|J| log |J|)$ from a sorting algorithm used in the algorithm. \par
%
%\begin{algorithm}
%\caption{Greedy Algorithm for Scheduling Problms}
%\label{algo:greedy}
%\begin{algorithmic}
%	\STATE Sort jobs by ending time
%	\STATE $S \gets \emptyset$
%	\WHILE {$A \neq \emptyset$}
%		\STATE $a_i \gets $ the job with the earliest end time in $A$.
%		\STATE $S \gets S \cup \{a_i\}$
%		\STATE $A \gets A - \{a_j | [s_j, f_j) \cap [s_i, f_i) \neq \emptyset\}$
%	\ENDWHILE
%	\RETURN $A$
%\end{algorithmic}
%\end{algorithm}
%
%In Subsection~\ref{subsec:algo-prob2}, where Problem 2 is discussed, this algorithm will be applied on multiple machines. In other words, some changes will be made to this algorithm to make it compatible to the airport scheduling problems which have more than one gate.

\section{Algorithms and Theory}
\label{sec:algo}

\subsection{Problem 1}
\label{subsec:algo-prob1}
We are asked to explore the flight dataset and make some observations. The result can be found in Subsection~\ref{subsec:expo-data}. \par
Next, the first problem asks us to find an algorithm for the simplified version of airport scheduling problems where the starting time and the ending time of each plane is the same. Here, we propose Algorithm~\ref{algo:p1}. \par

\begin{algorithm}
\caption{Simple Airport Scheduling} \label{algo:p1}
\begin{algorithmic}
\STATE $G \gets \emptyset$
\FORALL{$a_i$ in A}
	\STATE $g \gets $ an available gate in G at time, $s_i$
	\IF{There is no such $g$}
		\STATE $g \gets $ a new gate
		\STATE $G \gets G \cup \{g\}$
	\ENDIF
	\STATE $g$ is allocated $a_i$
\ENDFOR
\end{algorithmic}
\end{algorithm}

This algorithm can solve airport scheduling problems because all the planes are assigned to at least a gate. Time complexity is $O(|A|)$ if all procedures such as finding an available gate can be done in $O(1)$. We are going to prove that this algorithm gives the smallest number of required gates. \par
\begin{theorem}
Algorithm~\ref{algo:p1} uses $k$ gates where $k$ is the maximum number of overlapping starting time of all the planes.
\end{theorem}
\begin{proof}
Assume, for the sake of contradiction, that Algorithm~\ref{algo:p1} uses more than $k$ gates. Therefore, there must be at least $k+1$ gates. Let $a_i$ be the first plane assigned to the $(k+1)$\textsuperscript{th} gate. Therefore, before $a_i$ is selected, there are $k$ gates. In order for a new gate to be created, there must not be any available gates in $G$ at time $s_i$. Since a new gate is created, all $k$ gates must be occupied by some planes. Therefore, there are $k+1$ planes coming to the airport at $s_i$, which contradicts the definition of $k$.
\end{proof}
Next, we are going to show that this algorithm gives the optimal number of gates.
\begin{lemma}
The lower bound of the number of gates is $k$.
\end{lemma}
\begin{proof}
This can be proven by Dilworth's theorem as mentioned in Subsection~\ref{subsec:over-prel}. But we will use a simpler proof here. Suppose that the number of gates can be less than $k$. Therefore, the number of gates is less than or equal to $k-1$. Assume, for the sake of a contradiction, that $k-1$ is the possible number of gates. Let $t$ be when there are $k$ overlapping taking-off times of planes. Suppose that all $k-1$ planes have been assigned to gates. The $k^{th}$ plane cannot be scheduled to any gate as all the gates are occupied by earlier $k-1$ planes. Hence, $k-1$ is not a possible solution. We can use this prove for any number of planes less than $k$. Therefore, the number of gates must be greater than or equal to $k$.
%According to Dilworth's theorem, the maximum number of elements in any antichain, a set of elements no two of which are comparable to each other, equals to the minimum number of chains in any partition of the set into chains. In this case, the maximum number of overlapping starting time of all the planes is the maximum number of elements in any antichain. Therefore, the minimum number of chains which is the number of required gates is equal to $k$.
\end{proof}

\begin{theorem}
Algorithm~\ref{algo:p1} gives the optimal solution.
\end{theorem}
\begin{proof}
Let $OPT$ be the smallest number of gates required and $ANS$ be the number of gates used in Algorithm~\ref{algo:p1}. According to the previous lemma, $OPT \geq k$. But, $k$ is the number of required gates from this algorithm. Hence, $OPT \geq ANS$. In addition, $ANS \geq OPT$ because there cannot be any better solutions than the optimal solution. Hence, $ANS = OPT$.
\end{proof}


\subsection{Problem 2}
\label{subsec:algo-prob2}
In this problem, we are given an amount of time we need at a gate for a plane to take-off and arrive at the airport. Each plane needs 90 minutes before the scheduled and 30 minutes after the scheduled take-off time at a gate. Therefore, for a plane, $a_i$, taking off at time $t_i$, we can use interval $[s_i, f_i) = [t_i-90, t_i+30)$ to represent the time the plane needs at a gate. \par
We propose greedy algorithm, Algorithm~\ref{algo:p2}. The algorithm firstly sorts all the planes by starting time. Then, the algorithm iterates over all the planes. The algorithm assigns a plane to an empty gate. However, if there does not exist any empty gate at the start time of the plane, the algorithm will create a new gate and assign the gate to the plane. \par

\begin{algorithm}
\caption{Greedy Algorithm for Airport Scheduling}\label{algo:p2}
\begin{algorithmic}
\STATE \textbf{Sort} $A$ by starting time
\STATE $G \gets \emptyset$
\FORALL{$a_i$ in A}
	\STATE $g \gets $ a gate, $g \in G$, vacant at $s_i$.
	\IF{There is no such $g$}
		\STATE $g \gets$ a new gate \label{algo:p2:newGate}
		\STATE $G \gets G \cup \{g\}$
	\ENDIF
	\STATE schedule $a_i$ to $g$ \label{algo:p2:schedule}
\ENDFOR
\end{algorithmic}
\end{algorithm}

Next, we show that this algorithm works correctly. All the planes have a gate for taking off. In addition, all the gates can have at most 1 plane at a particular time. In addition, we also show that Algorithm~\ref{algo:p2} is optimal. In other words, the number of gates is minimum.\par

\begin{theorem}
All the planes have a gate, $g$, for taking-off. At a particular time, $t$, a gate has at most 1 plane.
\end{theorem}
\begin{proof}
Line~\ref{algo:p2:schedule} of Algorithm~\ref{algo:p2} schedules every plane. Therefore, all the planes must have a gate during its taking-off time. \par
Next, we show that a gate has at most 1 plane at a particular point of time, $t$. Assume, for the sake of contradiction, that there exists such a gate, $g$, that there are more than a plane at $t$. Therefore, there must be at least two planes at time $t$ at $g$ after executing this algorithm. Let $a_i$ and $a_j$ be the first plane and the second plane assigned to $g$. When $a_j$ is scheduled to $g$, $g$ cannot be a new gate from Line~\ref{algo:p2:newGate} because this gate was already scheduled to $a_i$. Therefore, the if-condition must be wrong. Hence, $g$ is a gate that is vacant at $s_j$. Because of the greedy property, $s_i < s_j$. In addition, $f_i \leq s_j$ because $g$ is empty at $s_j$. Hence, these two planes' intervals do not overlap and they cannot be assigned to any gate at any time. \par
Therefore, this algorithm can schedule aircraft to gates correctly.
\end{proof}

Let $OPT$ and $ANS$ be the optimal number of gates and the number of gates from the algorithm respectively. \par

\begin{lemma}
$ANS \geq OPT$ 
\end{lemma}
\begin{proof}
No other possible solutions have the fewer number of gates than the optimal solution.
\end{proof}

\begin{lemma}
Let the maximum number of overlapping intervals be $k$. $k$ is the lower bound of this problem.
\end{lemma}
\begin{proof}
Assume, for the sake of contradiction, the possible number of gates is less than $k$. Let time $t$ be when the maximum number of overlapping intervals occurs. Therefore, there must be a plane that is not scheduled to any gate as all the other gates are occupied. Hence, this solution is not possible.
\end{proof}

\begin{lemma}
$ANS = k$
\end{lemma}
\begin{proof}
Assume, for the sake of contradiction, $ANS \neq k$. Therefore, $ANS < k$ or $ANS > k$. \par
\underline{case 1} $ANS < k$: This is impossible because $k$ is the lower bound and $ANS$ is a valid solution. Therefore, $ANS \geq k$. \par
\underline{case 2} $ANS > k$: Therefore, $ANS$ must be at least $k+1$. Let $a_i$ be a plane that is firstly assigned to $g_{k+1}$. Therefore, there is no empty gate in $G$ at $s_i$ when $|G| = k$. Therefore, all $k$ gates have planes. Hence, there are $k+1$ planes overlapping at $s_i$. This is a contradiction because $k$ is the maximum number of overlapping intervals. \par
Therefore, $ANS = k$
\end{proof}

\begin{theorem}
The number of gates from Algorithm~\ref{algo:p2} is equal to the optimal number of gates.
\end{theorem}
\begin{proof}
$k$ is the lower bound and is equal to $ANS$. Therefore, $OPT \geq k = ANS$. We can conclude that $OPT = ANS$ because $OPT \geq ANS$ and $ANS \geq OPT$.
\end{proof}

After showing that the greedy algorithm is optimal, we use this algorithm on real dataset and the result is shown in Subsection~\ref{subsec:expo-resu}. \par

\subsubsection{Doubling the Number of Gates}
We also introduce one more interesting problem here: \textbf{do we need to double the number of gates to support twice as many flights?} It depends on how we increase the number of flights. Suppose that at first intervals $[0, t_1)$ and $[t_1, t_2)$ have the maximum number of overlapping plane taking-off intervals $2x$ and $x$ respectively. Therefore, we need $2x$ gates. Now we double the flights by adding $x$ and $2x$ plane intervals to time $[0, t_1)$ and $[t_1, t_2)$ respectively. Therefore, we will need at most $3x$ gates and $3x$ is not as many as twice of $2x$. \par
In addition, it is also possible that we need more than twice of the number of current gates. Again, assume that at first there are at most $x$ overlapping schedules in both intervals $[0, t_1)$ and $[t_1, t_2)$. We double the number of flights by adding $2x$ planes to the first interval and make those $2x$ planes cover time $t_a$, where there were previously $x$ overlapping intervals. Therefore, now there are $3x$ overlapping intervals. Hence, we need $3x$ gates which is more than two times of the previous number of gates required.

\subsection{Problem 3}
\label{subsec:algo-prob3}

\subsection{Problem 4}
\label{subsec:algo-prob4}

\section{Experiments}
\label{sec:expe}

\subsection{Methodology}
\label{subsec:expo-meth}
@TODO: DESCRIBE HOW WE SELECT DATASET AND THE REASON BEHIND

\subsubsection{Data Sets}
\label{subsec:expo-data}
@TODO: P1 DATASET ANALYSIS

\subsection{Main Results}
\label{subsec:expo-resu}
@TODO: P2 ANALYSIS OF RELATIONSHIP BETWEEN THE NUMBER OF FLIGHTS VS THE NUMBER OF GATES FROM THE REAL DATASET.

\section{Conclusions and Future Work}
\label{sec:conc}

\bibliographystyle{unsrt}
\bibliography{ref}


\end{document}